% Präsentationsvorlage Fionn Gantenbein
\documentclass[xcolor=dvipsnames]{beamer} %Option ist notwendig um mehr Farben
					  %zur Verfügung zu haben 

%Packages
\usepackage[ngerman]{babel}
\usepackage{amsmath}
\usepackage{lmodern}
\usepackage[T1]{fontenc}
\usepackage[utf8]{inputenc}
\usepackage{csquotes}
\usepackage{booktabs}

%%%%%%%%%%%%%%%%%%%%%%%%%%%%%%%%%%%%%%%%%%%%%%%%%%%%%%%%%%%%%%%%%%%%%%%%%%%%%%%%
%			  PERSONALIZE BEMAER
%%%%%%%%%%%%%%%%%%%%%%%%%%%%%%%%%%%%%%%%%%%%%%%%%%%%%%%%%%%%%%%%%%%%%%%%%%%%%%%%

%Farben
\setbeamercolor{topleft}{fg=black,bg=NavyBlue!63}
\setbeamercolor{bottommiddle}{fg=black,bg=blue!30}
\setbeamercolor{bottomleft}{fg=black,bg=NavyBlue!60}
\setbeamercolor{rightish}{fg=black,bg=blue!15}

%Bullets
\setbeamertemplate{itemize item}[circle]

%Kopfzeile
\setbeamertemplate{headline}
{\leavevmode
\begin{beamercolorbox}[ht=2.5ex,dp=1ex,left,wd=0.5\paperwidth]{topleft}
\hspace{2ex}\insertsection  
\end{beamercolorbox}%
\begin{beamercolorbox}[ht=2.5ex,dp=1ex,right,wd=0.5\paperwidth]{rightish}
\insertsubsection \hspace{2ex}
\end{beamercolorbox}
}

%Fusszeile
\setbeamertemplate{footline}
{\leavevmode
\begin{beamercolorbox}[ht=2.5ex,dp=1ex,left,wd=0.33333333333333333\paperwidth]{bottomleft}
\hspace{3ex}\insertauthor
\end{beamercolorbox}%
\begin{beamercolorbox}[ht=2.5ex,dp=1ex,wd=0.333333333333333333\paperwidth]{bottommiddle}
\end{beamercolorbox}%
\begin{beamercolorbox}[ht=2.5ex,dp=1ex,right,wd=0.333333333333333333\paperwidth]{rightish}
\insertshortdate \hspace{1mm} \insertframenumber/\inserttotalframenumber  \hspace{3ex}
\end{beamercolorbox}
}

%Keine Navigationsliste
\beamertemplatenavigationsymbolsempty

%Overlays
\setbeamercovered{still covered={\opaqueness<1->{0}},again covered={\opaqueness<1->{15}}}

%Title Autor Datum
\date{\today}
\title{Datenprojekt: \texttt{eulaw}}
\author{F. Preu, J. Körthner, F. Gantenbein}


%%%%%%%%%%%%%%%%%%%%%%%%%%%%%%%%%%%%%%%%%%%%%%%%%%%%%%%%%%%%%%%%%%%%%%%%%%%%%%%%%
%                                  PRESENTATION
%%%%%%%%%%%%%%%%%%%%%%%%%%%%%%%%%%%%%%%%%%%%%%%%%%%%%%%%%%%%%%%%%%%%%%%%%%%%%%%%%

\begin{document}
\frame{
\maketitle
}

\section{Einleitung}

\begin{frame}
  \frametitle{Zielsetzung}
  \begin{itemize}
  \item Datenbank des EU-Primärrechts
    \begin{itemize}
    \item Maschinenlesbar
    \item Reproduzierbar
    \end{itemize}
  \item Anwendungsbeispiel: Visualisierung der Entwicklung des EU-Primärrechts
  \end{itemize}
\end{frame}


\begin{frame}
  \frametitle{Weshalb hilft uns Jan Biesenbenders Arbeit nicht?}
  \begin{enumerate}
  \item Einheit in Jan Biesenbenders Arbeit sind \enquote{Veränderungen}
  \item Verträge können nicht wieder auf Artikelebene gebracht werden
  \item Reproduzierbarkeit: es bleibt unklar auf woher eine Veränderung kommt
  \end{enumerate}
\end{frame}

\section{Vorgehensweise}
\begin{frame}
  \frametitle{Grundidee}
  \begin{itemize}
  \item Wir halten die in den Verträgen festgehaltenen Änderungen fest und
   führen sie genau wie vorgeschrieben um. 
 \item Somit:
   \begin{enumerate}
   \item jede Veränderung nachvollziehbar
   \item zu jedem Zeitpunkt gültiges Primärrecht verfügbar
   \end{enumerate}
 \item Voraussetzungen:
   \begin{enumerate}
   \item Vertragstexte mit Referenznummer die Textstruktur abbildet
   \item Für jeden Änderungsvertrag einen Veränderungskatalog
   \end{enumerate}
  \end{itemize}
\end{frame}

\begin{frame}
  \frametitle{Texte \& Texquelle}
  \begin{itemize}
  \item Alle Verträge von \texttt{wikisource.org} 
    \begin{enumerate}
    \item Quelle mit deutschen, englischen Originaltexten stand
    \item Einfach in eine maschinell lesbare Form zu bringen
    \end{enumerate}
  \item Konkret: scrapen aller Treaties von \texttt{wikisource.org} und
    Nachbearbeiten
  \item Abbildung der Textstruktur in der Referenznummer
  \item Beispiel: Euratom-Vertrag
  \end{itemize}
\end{frame}

\begin{frame}
  \frametitle{Kodieren der Veränderungen}
  \begin{itemize}
  \item Schritte:
    \begin{enumerate}
    \item Sichtung der Verträge um die unterschiedlichen Anweisungen zu
      identifizieren
    \item Die Veränderungskataloge erstellen: Beispiel Merger-Vertag
    \item Veränderungen ausführen
    \end{enumerate}
  \end{itemize}
\end{frame}

\begin{frame}
  \frametitle{Text ersetzen}
  \begin{itemize}
  \item \texttt{replace}: einen Artikel mit Referenznummer \texttt{id}
    ersetzen.
  \item \texttt{replace\_txt}: in einem Artikel mit Referenznummer \texttt{id},
   Text finden und ersetzen.
 \item \texttt{replace\_txt\_globally}: in einem Vertrag mit Referenznummer
   \texttt{id} einen Ausdruck ersetzen.
  \end{itemize}
\end{frame}

\begin{frame}
  \frametitle{Text einfügen}
  \begin{itemize}
  \item \texttt{insert}: neuer Artikel mit Referenznummer \texttt{id} zu einem
    Vertrag hinzufügen
  \item \texttt{insert\_txt}: einen bestehenden Artikel mit Text ergänzen
  \item \texttt{add}: neuer, unabhängiger Gesetzestext mit Referenznummer
    \texttt{id} hinzufügen
  \end{itemize}
\end{frame}

\begin{frame}
  \frametitle{Weitere}
  \begin{itemize}
  \item \texttt{repeal}: Artikel aufheben
  \item \texttt{repeal\_txt}: Artikel teilweise aufheben
  \item \texttt{renumber}: Artikel neu nummerieren z.B. infolge eines neuen
    Kapitel      
  \item Beispiele: Änderungskatalog Merger-Vertag, endgültiger Datensatz.
  \end{itemize}
\end{frame}

\section{Ausblick}
\begin{frame}
  \frametitle{Was noch gemacht wird}
  \begin{itemize}
  \item Alles wird als \texttt{git} zur Verfügung gestellt
  \item Einfache Visualisierung, Beispiel
  \end{itemize}
\end{frame}
\end{document}
%EOF